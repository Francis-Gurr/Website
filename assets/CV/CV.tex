\documentclass[a4paper, 11pt, hidelinks]{article}

\usepackage[margin=0in]{geometry}
\usepackage[none]{hyphenat}
\usepackage{enumitem}
\usepackage{hyperref}


% Font
\usepackage[T1]{fontenc}
\usepackage[default]{sourcesanspro}
\usepackage{fontawesome}
\renewcommand{\baselinestretch}{1.05}

% Semi bold font
\newcommand{\textsb}[1]{
	\fontseries{sb}\selectfont #1 \normalfont
}

% Colour Pallete
\usepackage{xcolor}
\definecolor{text_color}{HTML}{121212} % Rich Black FOGRA 39
\definecolor{sidebar_color}{HTML}{c6ecec} %Powder Blue
\definecolor{bg_color}{HTML}{ffffff} % White
\pagecolor{bg_color}
\color{text_color}

% Section Headings
\usepackage{titlesec}
\titleformat{\section}[block]{\LARGE\bfseries\center}{}{0em}{}
\titlespacing*{\section}{0em}{0em}{0.25em}

% Experience page headings \xp{Name}{Place}{Date}
\newcommand{\xp}[3]{
	\vspace{0.25em}
	\textbf{\large#1} \quad \emph{\large#2} \hfill \textbf{\large#3}
}

% Skill boxes
\usepackage[skins]{tcolorbox}
\newcommand{\skill}[1]{
	\tcbox[skin=enhanced, nobeforeafter, colback=bg_color, colframe=bg_color, coltext=text_color, size=fbox, height=15pt]{\footnotesize#1}
}

% Custom itemise
\usepackage{xparse}
\NewDocumentEnvironment{itemise}{O{1em} O{-0.25em} O{-0.5em} O{0em}}
{\begin{itemize}[align=parleft, left=0pt..{#1}]
\setlength\itemsep{#2}
\vspace{#3}}
{\vspace{#4}
\end{itemize}}

%No page numbering
\pagenumbering{gobble}

\begin{document}
% Sidebar
\noindent\fcolorbox{sidebar_color}{sidebar_color}{%
	\color{text_color}
	\begin{minipage}{0.33\textwidth}
		\hspace{0.063\textwidth}
		\begin{minipage}{0.87\textwidth}
			\vspace{1.5em}
			\section{\huge Francis Gurr}

			% CONTACT
			\begin{itemise}[2em][0em][0.5em][0em]
				\item[\Large \faHome] Sheffield, UK
				\item[\Large \faEnvelope] \href{mailto:francis.gurr@gmail.com}{francis.gurr@gmail.com}
				\item[\Large \faGlobe] \href{https://www.francisgurr.com}{francisgurr.com}
				\item[\Large \faLinkedin] \href{https://www.linkedin.com/in/francis-gurr/}{francis-gurr}
				\item[\Large \faGithub] \href{https://github.com/Francis-Gurr}{Francis-Gurr}
			\end{itemise}

			% SUMMARY
			\section{Summary}
			I am a recent Electronic Engineering (MEng) graduate from  Durham University.

			I enjoy solving complex problems both independently and as part of a team.
			I am a fast learner and enjoy widening my skill set by challenging myself with personal projects.
			I have experience working on large long term projects and can work well in a range of team dynamics.
			Whilst completing projects I am highly motivated, organised and strive to ensure all my work is of a high standard.
			These attributes have led me to receive a first in every programming assignment throughout my degree.

			In my spare time I am a keen rock climber, mountain unicyclist and occasional juggler.

			% SKILLS
			\section{Skills}
			\emph{Most experience with:\vspace{0.25em}}\\
			\skill{Java}
			\skill{Python}
			\skill{C}
			\skill{Javascript}
			\skill{MatLab}

			\vspace{0.25em}

			\skill{Electronics}
			\skill{Linux}
			\skill{LaTeX}
			\skill{German}
			\vspace{0.5em}
			\\
			\emph{Some experience with:\vspace{0.2em}}\\
			\skill{C++}
			\skill{React}
			\skill{Express.js}
			\skill{Node.js}

			\vspace{0.25em}

			\skill{MySQL}
			\skill{PHP}
			\skill{Graphic Design}

			% REFERENCES
			\section{References}
			\textbf{Dr Stefano Giani}
			\begin{itemise}[2em][-0.25em][-0.5em][0em]
				\item[\faBriefcase] Assistant Professor\\ \emph{Durham University}
				\item[\faEnvelope] \href{mailto:stefano.giani@durham.ac.uk}{stefano.giani@durham.ac.uk}
			\end{itemise}

			\textbf{Colin Reekie}
			\begin{itemise}[2em][-0.25em][-0.5em][0em]
				\item[\faBriefcase] Head of Development\\ \emph{Q-Free ASA}
				\item[\faEnvelope] \href{mailto:colin.reekie@q-free.com}{colin.reekie@q-free.com}
			\end{itemise}
			\vfill
			\vspace{2.5em}
		\end{minipage}
	\end{minipage}
}
% Spacing
\hspace{0.02\textwidth}
% Main body
\begin{minipage}{0.587\textwidth}
	\vspace{-0.25em}

	% EXPERIENCE
	\section{Experience}
	% Final Year Project
	\xp{Masters Project - 1st Class (80\%)}{Durham University}{2019 - 2020}
	\begin{itemise}

		%Q-Free, who are developing a vehicle classification system using video images.
		%This project investigated the use of video cameras to determine the speed of detected vehicles.
		%This could add another unique product to the ITS market, providing clients with the flexibility to choose the data they require.
		\item Q-Free, a global leader in intelligent transportation systems (ITS), offered me the opportunity to begin developing a new product for my masters project.
		\item My project proposed the use of video images to determine the speed of detected vehicles. As a non-intrusive alternative to current ITS and infomobility systems, the aim was to add a unique product to the market.
		\item I used a convolutional neural network for object detection with an accuracy of 98\% mAP and a Kalman filter was used to track the vehicles.
		\item I developed Python software to calibrate the camera using road markings, and used C++ to calculate the vehicle speeds.
		\item The resulting software was able to provide vehicle speeds in real-time from road-side camera footage.
		\hfill
		\href{https://www.francisgurr.com/pages/masters_project/masters_project.html}{\faLink}
	\end{itemise}
	% Internship
	\xp{R\&D Internship}{Q-Free ASA, Bristol}{Jul 2019 - Sep 2019}
	\begin{itemise}
		\item During my third year design project, I designed an innovative prototype for a non-intrusive roadside detection system to count and classify vehicles.
		\item Q-Free awarded me the opportunity to develop a working prototype of my design during a summer internship.
		\item I developed software in C to process data from LiDAR and radar sensors, and used Python to generate graphical 2D side profiles of each vehicle in real time.
		\hfill
		\href{https://www.francisgurr.com/pages/qfree_internship/qfree_internship.html}{\faLink}
	\end{itemise}
	% Co-author
	\xp{Internship}{Durham University, Maths Dept.}{Jun 2018 - Feb 2019}
	\begin{itemise}
		\item Co-authored ancillary software for an academic paper entitled \emph{Quartic Graphs that are Bakry-Émery Curvature Sharp}, published in Discrete Mathematics \textsb{343}(3), \href{https://arxiv.org/abs/1902.10665}{DOI: 10.1016/j.disc.2019.111767}.
		\item I developed a computer classification algorithm in Python to recursively generate all unique radius two local configurations of quartic graphs.
		\item These results were the basis of the main theorem in an academic research paper.
		\hfill
		\href{https://www.francisgurr.com/pages/summer_project/summer_project.html}{\faLink}
	\end{itemise}

	% EDUCATION
	\vspace{-2em}
	\section{Education}
	% MEng
	\xp{MEng Electronic Engineering - 2:1}{Durham University}{2015 - 2020}
	\begin{itemise}
		\item Took a year out following bereavement as an exam only student.
	\end{itemise}
	% MChem
	\xp{MChem Chemistry}{Durham University}{2014 - 2015}
	\begin{itemise}
		\item Switched course after year one.
	\end{itemise}
	% A Levels
	\xp{A Levels}{Bournemouth Grammar School}{2013 - 2014}
	\begin{itemise}
		\item A* Chemistry, A Maths, A Physics, A German.
	\end{itemise}

	% ACHIEVEMENTS
	\section{Personal Projects}
	\xp{Pathfinding Visualiser}{}{2021 - Ongoing}
	\begin{itemise}
		\item Currently working on a full stack web application to visually depict pathfinding algorithms in action with real world map data.
		\item \textsb{Skills used:} Javascript, React, Express.js, Node.js, MySQL.
		\hfill
		\href{https://github.com/Francis-Gurr/Pathfinding-Visualiser}{\faLink}
	\end{itemise}
	\xp{Sudoku Solver}{}{Summer 2017}
	\begin{itemise}
		\item Created a JavaFX app to solve Sudoku problems using a recursive backtracking algorithm.
		\item \textsb{Skills used:} Java, XML.
		\hfill
		\href{https://github.com/Francis-Gurr/Sudoku-Solver}{\faLink}
	\end{itemise}
\end{minipage}
\end{document}
